\section{Vector Storage Schemes}


  \paragraph{Class}{\tt \hypertarget{Array}{Array<T>}}
  \index{Array}


% template parameter
  \begin{CDescription}
  \item[T]
      Template parameter specifying the array's element type.
  \end{CDescription}

% typedefs
  \paragraph{{\it Public typedefs}}
  \hypertarget{Array:typedefs}{}
  
  \begin{CDescription}
  \item[typedef ArrayView<T> View;           ,,%
        typedef ConstArrayView<T> ConstView;]
      Vector classes using engines and supporting views have to determine
      statically the class realizing views:
      \begin{lstlisting}
      template <typename Engine>
      class SomeVector
      {
          ...
          
          // method to create view
          typename E::View
          createView(int from, int to);

          ...    
      };
      \end{lstlisting}
      For a realistic example see \hyperlink{DenseVector}{{\tt DenseVector<T>}}.
  \end{CDescription}

% constructors and destructor
  \paragraph{{\it Constructors and destructor}}
  \begin{CDescription}
  \item[Array()]
      Creates an empty array.  No memory gets allocated.  Length of the vector
      is zero.    
  
  \item[Array(int length, int firstIndex=1);]
      Vectors can be created by a constructor taking two arguments.
      The first argument specifies the length the second the base for indexing.
      The second argument a default value of one such that vectors by default
      are indexed one-based like in FORTRAN. 

  \item[Array(const Array<T> &rhs);             ,,%
        Array(const ArrayView<T> &rhs);         ,,%
        Array(const ConstArrayView<T> &rhs);]
      Copy constructor and constructors to initializing the Array with
      a copy of an existing array.

  \item[~Array()]
      Detaches the reference counter for the allocated memory.  If no view
      references the allocated any more the memory gets released.
      
\end{CDescription}

  \paragraph{{\it Operators}}
  \begin{CDescription}
  \item[Array<T> &                              ,,%
        operator=(const Array<T>  &rhs);        ,,%
                                                ,,%
        Array<T> &                              ,,%
        operator=(const ArrayView<T> &rhs);     ,,%
                                                ,,% 
        Array<T> &                              ,,%
        operator=(const ConstArrayView<T> &rhs);]
      The array gets overwritten by the left hand side array. If lengths do
      not match the left hand side gets resized by calling {\tt resize}. 

\item[const T &                               ,,%
      operator()(int index) const;            ,,%
                                              ,,%
      T &                                     ,,%
      operator()(int index);]
   Returns reference of array element with index {\tt index}.

   \emph{Debug-Mode:} Check if {\tt index} is in the range 
                      {\tt firstIndex} \dots {\tt lastIndex}.
\end{CDescription}

%methods
  \paragraph{{\it Methods}}
  \begin{CDescription}
  \item[int ,,%
        firstIndex() const;]
      Returns index of first array element.

  \item[int ,,%
        lastIndex() const;]
      Returns index of last array element.

  \item[int ,,%
        length() const;]
      Returns array length.

  \item[int ,,%
        stride() const;]
      Returns stride in memory between array elements.

  \item[const T *           ,,%
        data() const;       ,,%
                          ,,%
        T *                 ,,%
        data();]
      Returns a pointer to the first array element (i.e., the element at index
      {\tt firstIndex}).

  \item[void ,,%
        resize(int length, int firstIndex = 1);]
      Resizes the array to store {\tt length} elements.  Indexing is based on
      {\tt firstIndex}.
      
      If no views sharing the same data do exist the previously allocated memory
      gets deallocated.
        
  \item[ConstArrayView<T>                                                 ,,%
        view(int from, int to,                                            ,,%
        . . int stride=1, int firstViewIndex=1) const;                    ,,%
                                                                          ,,%
        ArrayView<T>                                                      ,,%
        view(int from, int to,                                            ,,%
         . . int stride=1, int firstViewIndex=1);]

      Creates and returns an array view.  The view references elements
      with indices {\tt from}, {\tt from + stride}, \dots, {\tt to} of the
      original array.
    
      Indexing of the returned view is based on {\tt firstViewIndex}.
  \end{CDescription}
  
  
\newpage
  


  \paragraph{Class}{\tt \hypertarget{ArrayView}{ArrayView<T>}}
  \index{Array!ArrayView}


% template parameter
  \begin{CDescription}
  \item[T]
      Template parameter specifying the array's element type.
  \end{CDescription}

% typedefs
  \paragraph{{\it Public typedefs}}
  \begin{CDescription}
  \item[typedef ArrayView<T> View;           ,,%
        typedef ConstArrayView<T> ConstView;]
      See \hyperlink{Array:typedefs}{{\it public typedefs of} {\tt Array<T>}}.
  \end{CDescription}

% constructors and destructor
  \paragraph{{\it Constructors and destructor}}
  \begin{CDescription}
  \item[ArrayView(void *storage, T *data,
                  int length, int stride=1, int firstIndex=1);]
          
  
  \item[ArrayView(const ArrayView &rhs);]

  \item[~ArrayView()]
      Detaches the reference counter for the allocated memory.  If no view
      references the allocated any more the memory gets released.
      
\end{CDescription}

  \paragraph{{\it Operators}}
  \begin{CDescription}
  \item[ArrayView<T> &                          ,,%
        operator=(const Array<T>  &rhs);        ,,%
                                                ,,%
        ArrayView<T> &                          ,,%
        operator=(const ArrayView<T> &rhs);     ,,%
                                                ,,% 
        ArrayView<T> &                          ,,%
        operator=(const ConstArrayView<T> &rhs);]
      The array gets overwritten by the left hand side array. 
      
      \emph{Note:} Unlike \hyperlink{Array}{Array} objects an {\tt ArrayView}
      object can not be resized (as this would imply that views can allocate
      their own memory).
      
      \emph{Debug-Mode}: Check if length matches length of the right hand side. 
      
\item[const T &                               ,,%
      operator()(int index) const;            ,,%
                                              ,,%
      T &                                     ,,%
      operator()(int index);]
   Returns reference of array element with index {\tt index}.

   \emph{Debug-Mode:} Check if {\tt index} is in the range 
                      {\tt firstIndex} \dots {\tt lastIndex}.
\end{CDescription}

%methods
  \paragraph{{\it Methods}}
  \begin{CDescription}
  \item[int ,,%
        firstIndex() const;]
      Returns index of first array element.

  \item[int ,,%
        lastIndex() const;]
      Returns index of last array element.

  \item[int ,,%
        length() const;]
      Returns array length.

  \item[int ,,%
        stride() const;]
      Returns stride in memory between array elements.

  \item[const T *           ,,%
        data() const;       ,,%
                          ,,%
        T *                 ,,%
        data();]
      Returns a pointer to the first array element (i.e., the element at index
      {\tt firstIndex}).
        
  \item[ConstArrayView<T>                                                 ,,%
        view(int from, int to,                                            ,,%
        . . int stride=1, int firstViewIndex=1) const;                    ,,%
                                                                          ,,%
        ArrayView<T>                                                      ,,%
        view(int from, int to,                                            ,,%
         . . int stride=1, int firstViewIndex=1);]

      Creates and returns an array view.  The view references elements
      with indices {\tt from}, {\tt from + stride}, \dots, {\tt to} of the
      original array.
    
      Indexing of the returned view is based on {\tt firstViewIndex}.
  \end{CDescription}
  
  
\newpage
  


  \paragraph{Class}{\tt \hypertarget{ConstArrayView}{ConstArrayView<T>}}
  \index{Array!ConstArrayView}

% template parameter
  \begin{CDescription}
  \item[T]
      Template parameter specifying the array's element type.
  \end{CDescription}

% typedefs
  \paragraph{{\it Public typedefs}}
  \begin{CDescription}
  \item[typedef ArrayView<T> View;           ,,%
        typedef ConstArrayView<T> ConstView;]
      See \hyperlink{Array:typedefs}{{\it public typedefs of} {\tt Array<T>}}.
  \end{CDescription}

% constructors and destructor
  \paragraph{{\it Constructors and destructor}}
  \begin{CDescription}
  \item[ConstArrayView(void *storage, T *data,
                  int length, int stride=1, int firstIndex=1);]
          
  
  \item[ConstArrayView(const ArrayView &rhs);]

  \item[~ArrayView()]
      Detaches the reference counter for the allocated memory.  If no view
      references the allocated any more the memory gets released.
      
\end{CDescription}

% operators
  \paragraph{{\it Operators}}
  \begin{CDescription}
  \item[const T &                               ,,%
        operator()(int index) const;]
      Returns reference of array element with index {\tt index}.

      \emph{Debug-Mode:} Check if {\tt index} is in the range 
                         {\tt firstIndex} \dots {\tt lastIndex}.
\end{CDescription}

%methods
  \paragraph{{\it Methods}}
  \begin{CDescription}
  \item[int ,,%
        firstIndex() const;]
      Returns index of first array element.

  \item[int ,,%
        lastIndex() const;]
      Returns index of last array element.

  \item[int ,,%
        length() const;]
      Returns array length.

  \item[int ,,%
        stride() const;]
      Returns stride in memory between array elements.

  \item[const T *           ,,%
        data() const;       ,,%
                          ,,%
        T *                 ,,%
        data();]
      Returns a pointer to the first array element (i.e., the element at index
      {\tt firstIndex}).
        
  \item[ConstArrayView<T>                                                 ,,%
        view(int from, int to,                                            ,,%
        . . int stride=1, int firstViewIndex=1) const;]

      Creates and returns an array view.  The view references elements
      with indices {\tt from}, {\tt from + stride}, \dots, {\tt to} of the
      original array.
    
      Indexing of the returned view is based on {\tt firstViewIndex}.
  \end{CDescription}
